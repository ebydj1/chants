% !TEX TS-program = lualatex
% !TEX encoding = UTF-8

\documentclass[11pt]{article} % use larger type; default would be 10pt

\usepackage{fontspec}
\usepackage{graphicx} % support the \includegraphics command and options
\usepackage{geometry} % See geometry.pdf to learn the layout options. There are lots.
\geometry{letterpaper}
\usepackage[autocompile]{gregoriotex} % for gregorio score inclusion
\usepackage{fancyhdr}
\usepackage{fullpage}
\usepackage{scrextend}
\usepackage{paracol}

% This defines a formatting type for all _m_ain _ref_erences
\newcommand{\mref}{\textbf}
% This defines a formatting type for all _v_erse _ref_erences
\newcommand{\vref}{\emph}
% This defines a constant environment for translations
\newenvironment{translation}
  {\begin{addmargin}{20pt}}
  {\end{addmargin}}
% This defines a constant environment for headings
\newenvironment{heading}
  {\begin{center}\begin{em}}
  {\end{em}\end{center}}

\pagestyle{fancy}
\fancyhf{}
\chead{\textsc{Pro Sponso et Sponsa}}
\cfoot{\thepage}
\renewcommand{\headrulewidth}{0pt}
\setlength{\headheight}{13.6pt}
\setlength{\headsep}{6pt}
\setlength{\columnseprule}{0.2pt}
\colseprulecolor{gray}
\grechangedim{annotationraise}{0pt}{scalable}

\begin{document}

\thispagestyle{plain}

\begin{center}\begin{huge}
  \textsc{Pro Sponso et Sponsa}
\end{huge}\end{center}

\gregorioscore{in--deus_israel--solesmes}
\begin{translation}
  \mref{Tob.~7:15, 8:19.} May the God of Israel join you together, and may he be
  with you, who was merciful to two only children: and now, O Lord, make them
  bless thee more fully. \vref{Ps.~127:1.} Blessed are all they that fear the
  Lord, that walk in his ways.
\end{translation}

\begin{heading}Collect\end{heading}

\begin{paracol}{2}
  \noindent Exaudi nos, omnipotens et misericors Deus: ut, quod nostro
  ministratur officio, tua benedictione potius impleatur.

  \switchcolumn

  \noindent Hear us, almighty and merciful God: that what is performed by our
  ministry, may be abundantly fulfilled with thy blessing.
\end{paracol}

\begin{heading}Epistle: Eph.~5:22-33.\end{heading}

\begin{paracol}{2}
  \noindent Fratres: Mulieres viris suis subditæ sint, sicut Domino: quoniam vir
  caput est mulieris, sicut Christus caput est Ecclesiæ: ipse, salvator corporis
  ejus.  Sed sicut Ecclesia subjecta est Chris\-to, ita et mulieres viris suis
  in omnibus.
  
  Viri, diligite uxores vestras, sicut et Christus dilexit Ecclesiam, et seipsum
  tradidit pro ea, ut illam sanctificaret, mundans lavacro aquæ in verbo vitæ,
  ut exhiberet ipse sibi gloriosam Ecclesiam, non habentem maculam, aut rugam,
  aut aliquid hujusmodi, sed ut sit sancta et immaculata. Ita et viri debent
  diligere uxores suas ut corpora sua. Qui suam uxorem diligit, seipsum diligit.
  Nemo enim umquam carnem suam odio habuit: sed nutrit et fovet eam, sicut et
  Christus Ecclesiam: quia membra sumus corporis ejus, de carne ejus et de
  ossibus ejus.  Propter hoc relinquet homo patrem et matrem suam, et adhærebit
  uxori suæe, et erunt duo in carne una. Sacramentum hoc magnum est, ego autem
  dico in Christo et in Ecclesia.
  
  Verumtamen et vos singuli, unusquisque uxorem suam sicut seipsum diligat: uxor
  autem timeat virum suum. 

  \switchcolumn

  \noindent Brethren: Let women be subject to their husbands, as to the Lord:
  Because the husband is the head of the wife, as Christ is the head of the
  church. He is the saviour of his body. Therefore as the church is subject to
  Christ, so also let the wives be to their husbands in all things.
  
  Husbands, love your wives, as Christ also loved the church, and delivered
  himself up for it: That he might sanctify it, cleansing it by the laver of
  water in the word of life: That he might present it to himself a glorious
  church, not having spot or wrinkle, or any such thing; but that it should be
  holy, and without blemish. So also ought men to love their wives as their own
  bodies. He that loveth his wife, loveth himself. For no man ever hated his own
  flesh; but nourisheth and cherisheth it, as also Christ doth the church:
  Because we are members of his body, of his flesh, and of his bones. For this
  cause shall a man leave his father and mother, and shall cleave to his wife,
  and they shall be two in one flesh. This is a great sacrament; but I speak in
  Christ and in the church.
  
  Nevertheless let every one of you in particular love his wife as himself: and
  let the wife fear her husband. 
\end{paracol}

\vskip10pt

\gregorioscore{gr--uxor_tua--solesmes}
\begin{translation}
  \mref{Ps.~127:3.} Thy wife shall be as a fruitful vine on the sides of thy
  house. \Vbar{}. Thy children as olive plants round about thy table.
\end{translation}

\vskip10pt

\gregorioscore{al--mittat_vobis--solesmes}
\begin{translation}
  \mref{Mt.~11:28.} Come to me all you that labour and are heavy laden, and I
  will refresh you.
\end{translation}

\pagebreak

\begin{heading}Gospel: Mt.~19:3-6.\end{heading}

\begin{paracol}{2}
  \noindent In illo tempore: Accesserunt ad eum pharisæi tentantes eum, et
  dicentes: Si licet homini dimittere uxorem suam, quacumque ex causa? Qui
  respondens, ait eis: Non legistis, quia qui fecit hominem ab initio, masculum
  et feminam fecit eos? Et dixit: Propter hoc dimittet homo patrem, et matrem,
  et adhærebit uxori suæ, et erunt duo in carne una. Itaque jam non sunt duo,
  sed una caro. Quod ergo Deus conjunxit, homo non separet. 

  \switchcolumn

  \noindent At that time: There came to him the Pharisees tempting him, and
  saying: Is it lawful for a man to put away his wife for every cause? Who
  answering, said to them: Have ye not read, that he who made man from the
  beginning, made them male and female? And he said: For this cause shall a man
  leave father and mother, and shall cleave to his wife, and they two shall be
  in one flesh. Therefore now they are not two, but one flesh. What therefore
  God hath joined together, let no man put asunder. 
\end{paracol}

\gregorioscore{of--in_te_speravi--solesmes}
\begin{translation}
  \mref{Ps.~30:15-16.} In thee, O Lord, have I hoped: I said: Thou art my God,
  my times are in thy hands.
\end{translation}

\begin{heading}Secret\end{heading}

\begin{paracol}{2}
  \noindent Suscipe, quæsumus, Domine, pro sacra connubii lege munus oblatum:
  et cujus largitor es operis, esto dispositor.

  \switchcolumn

  \noindent Accept, we beseech thee, O Lord, the gifts offered for the sacred
  law of marriage: and do thou direct the work which thou didst establish.
\end{paracol}

\gregorioscore{co--ecce_sic_benedicetur--solesmes}
\begin{translation}
  \mref{Ps.~127:4-6.} Behold, thus shall every man be blessed that feareth the
  Lord; and mayest thou see thy children's children; peace upon Israel.
\end{translation}

\begin{heading}Postcommunion\end{heading}

\begin{paracol}{2}
  \noindent Quæsumus, omnipotens Deus: instituta providentiæ tuæ pio favore
  comitare; ut quos legitima societate connectis, longæva pace custodias.

  \switchcolumn

  \noindent We beseech thee, almighty God, to accompany with thy gracious favour
  the institution of thy providence, and keep in lasting peace those whom thou
  dost join in lawful union.
\end{paracol}

\end{document}
