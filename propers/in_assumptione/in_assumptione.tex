% !TEX TS-program = lualatex
% !TEX encoding = UTF-8

\documentclass[11pt]{article} % use larger type; default would be 10pt

% usual packages loading:
\usepackage{fontspec}
\usepackage{graphicx} % support the \includegraphics command and options
\usepackage{geometry} % See geometry.pdf to learn the layout options. There are lots.
\geometry{letterpaper} % or letterpaper (US) or a5paper or....
\usepackage[autocompile]{gregoriotex} % for gregorio score inclusion
\usepackage{scrextend}
\usepackage{paracol}

\newcommand{\mref}{\textbf}
\newcommand{\vref}{\emph}
\newenvironment{translation}
  {\begin{addmargin}{20pt}}
  {\end{addmargin}}

\begin{document}

\begin{center}\begin{huge}
  \textsc{In Assumptione Beatae Mariae Virginis}
\end{huge}\end{center}

\gregorioscore{in--signum_magnum--solesmes}
\begin{translation}
  \mref{Apoc. 12. 1.} A great sign appeared in heaven: a woman clothed with the
  sun, and the moon under her feet, and on her head a crown of twelve stars.
  \vref{Ps. 97. 1.} Sing ye to the Lord a new canticle: because He hath done
  wonderful things.
\end{translation}

\begin{center}\emph{Collect}\end{center}

\begin{paracol}{2}
  \noindent Omnipotens sempiterne Deus, que Immaculatam Virginem Mariam, Filii
  tui Genitricem, corpore et anima ad cælestem gloriam assumpsisti: concede,
  quæsumus; ut ad superna semper intenti, ipsius gloriæ mereamur esse consortes.

  \switchcolumn
  
  \noindent Almighty, everlasting God, who took up, body and soul, the
  Immaculate Virgin Mary, Mother of your Son, into heavenly glory, grant, we
  beseech you, that, always devoting ourselves to heavenly things, we may be
  found worthy to share in her glory.
\end{paracol}

\begin{center}\emph{Lesson: Judith 13, 22-25; 15, 10.}\end{center}

\begin{paracol}{2}
  \noindent Benedixit te Dominus in virtute sua, quia per te ad nihilum redegit
  inimicos nostros. Benedicta es tu, filia, a Domino Deo excelso, præ omnibus
  mulieribus super terram. Benedictus Dominus, qui creavit cælum et terram, qui
  te direxit in vulnera capitis principis inimicorum nostrorum; quia hodie nomen
  tuum ita magnificavit, ut non recedat laus tua de ore hominum, qui memores
  fuerint virtutis Domini in æternum, pro quibus non pepercisti animæ tuæ
  propter angustias et tribulationem generis tui, sed subvenisti ruinæ ante
  conspectum Dei nostri. Tu gloria Jerusalem, tu lætitia Israel, tu
  honorificentia populi nostri.

  \switchcolumn

  \noindent The Lord hath blessed thee by his power, because by thee he hath
  brought our enemies to nought. Blessed art thou, O daughter, by the Lord the
  most high God, above all the women upon the earth. Blessed be the Lord who
  made heaven and earth, who hath directed thee to the cutting off the head of
  the prince of our enemies. Because he hath so magnified thy name this day,
  that thy praise shall not depart out of the mouth of men who shall be mindful
  of the power of the Lord for ever, for that thou hast not spared thy life, by
  reason of the distress and tribulation of thy people, but hast prevented our
  ruin in the presence of our God. Thou art the glory of Jerusalem, thou art the
  joy of Israel, thou art the honour of our people.
\end{paracol}

\vskip10pt

\gregorioscore{gr--audi_filia_et_concupiscet--solesmes}
\begin{translation}
  \mref{Ps.~44.~11, 12, 14.} Hearken, O daughter, and see, and incline thy ear.
  And the king shall greatly desire thy beauty. \Vbar{}. All the glory of the
  king's daughter is within in golden borders.
\end{translation}

\vskip20pt

\gregorioscore{al--assumpta_est--solesmes}
\begin{translation}
  Mary is taken up into heaven: the host of Angels rejoice.
\end{translation}

\vskip10pt

\begin{center}\emph{Gospel: Luc.~1, 41-50.}\end{center}

\begin{paracol}{2}
  \noindent In illo tempore: Repleta est Spiritu Sancto Elisabeth et exclamavit
  voce magna, et dixit: Benedicta tu inter mulieres, et benedictus fructus
  ventris tui. Et unde hoc mihi ut veniat mater Domini mei ad me? Ecce enim ut
  facta est vox salutationis tuæ in auribus meis, exsultavit in gaudio infans in
  utero meo. Et beata, quæ credidisti, quoniam perficientur ea, quæ dicta sunt
  tibi a Domino. Et ait Maria: Magnificat anima mea Dominum; et exsultavit
  spiritus meus in Deo salutari meo; quia respexit humilitatem ancillæ suæ, ecce
  enim ex hoc beatam me dicent omnes generationes. Quia fecit mihi magna qui
  potens est, et sanctum nomen ejus, et misericordia ejus a progenie in
  progenies timentibus eum.

  \switchcolumn

  \noindent In that time: Elizabeth was filled with the Holy Ghost, and she
  cried out with a loud voice, and said: Blessed art thou among women, and
  blessed is the fruit of thy womb. And whence is this to me, that the mother of
  my Lord should come to me? For behold as soon as the voice of thy salutation
  reached my ears, the infant in my womb leaped for joy. And blessed art thou
  that hast believed, because those things shall be accomplished that were
  spoken to thee by the Lord. And Mary said: My soul doth magnify the Lord. And
  my spirit hath rejoiced in God my Saviour. Because he hath regarded the
  humility of his handmaid; for behold from henceforth all generations shall
  call me blessed. Because he that is mighty, hath done great things to me; and
  holy is his name. And his mercy is from generation unto generations, to them
  that fear him.
\end{paracol}

\vskip20pt

\gregorioscore{of--inimicitias_ponam--solesmes}
\begin{translation}
  \mref{Gen. 3. 15.} I will put enmities between thee and the woman, and thy
  seed and her seed.
\end{translation}

\vskip10pt

\begin{center}\emph{Secret}\end{center}

\begin{paracol}{2}
  \noindent Ascendat ad te, Domine, nostræ devotionis oblatio, et, Beatissima
  Virgine Maria in cælum assumpta intercedente, corda nostra, caritatis igne
  successa, ad te jugiter adspirent.

  \switchcolumn

  \noindent May the offering of our devotion rise unto you, O Lord, and by the
  intercession of the Most Blessed Virgin Mary, who was taken up into heaven,
  may our hearts, on fire with love, strive ever upward to you.
\end{paracol}

\vskip10pt

\gregorioscore{co--beatam_me_dicent--solesmes}
\begin{translation}
  \mref{Lk. 1. 48-49.} All generations shall call me blessed, because He that is
  mighty hath done great things to me.
\end{translation}

\vskip10pt

\begin{center}\emph{Postcommunion}\end{center}

\begin{paracol}{2}
  \noindent Sumptis, Domine, salutaribus sacramentis, da, quæsumus, ut, meritis
  et intercessione Beatæ Virginis Mariæ in cælum assumptæ, ad resurrectionis
  gloriam perducamur.

  \switchcolumn

  \noindent Having partaken, O Lord, of the sacrament of salvation, grant, we
  beseech you, that through the merits and intercession of the Blessed Virgin
  Mary, who was taken up into heaven, we may be delivered from all the evils
  that threaten us.
\end{paracol}

\end{document}
