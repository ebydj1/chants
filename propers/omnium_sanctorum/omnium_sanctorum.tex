% !TEX TS-program = lualatex
% !TEX encoding = UTF-8

\documentclass[11pt]{article} % use larger type; default would be 10pt

% usual packages loading:
\usepackage{fontspec}
\usepackage{graphicx} % support the \includegraphics command and options
\usepackage{geometry} % See geometry.pdf to learn the layout options. There are lots.
\geometry{letterpaper} % or letterpaper (US) or a5paper or....
\usepackage[autocompile]{gregoriotex} % for gregorio score inclusion
\usepackage{fancyhdr}
\usepackage{fullpage}
\usepackage{scrextend}
\usepackage{paracol}

\newcommand{\mref}{\textbf}
\newcommand{\vref}{\emph}
\newenvironment{translation}
  {\begin{addmargin}{20pt}}
  {\end{addmargin}}

\pagestyle{fancy}
\fancyhf{}
\chead{\textsc{In Festo Omnium Sanctorum}}
\cfoot{\thepage}
\renewcommand{\headrulewidth}{0pt}
\setlength{\headheight}{13.6pt}
\setlength{\headsep}{6pt}

\begin{document}

\thispagestyle{plain}

\begin{center}\begin{huge}
  \textsc{In Festo Omnium Sanctorum}
\end{huge}\end{center}

\grechangedim{annotationraise}{0pt}{scalable}
\gregorioscore{in--gaudeamus_sanctorum_omnium--solesmes}
\begin{translation}
  Let us all rejoice in the Lord, celebrating a festival-day in honour of all
  the saints: at whose solemnity the angels rejoice, and give praise to the Son
  of God. \vref{Ps.~32:1.} Rejoice in the Lord, ye just: praise becometh the
  upright.
\end{translation}

\begin{center}\emph{Collect}\end{center}

\setlength{\columnseprule}{0.2pt}
\colseprulecolor{gray}

\begin{paracol}{2}
  \noindent Omnipotens sempiterne Deus, qui nos omnium Sanctorum tuorum merita
  sub una tribuisti ce\-le\-bri\-ta\-te venerari: quæsumus; ut desideratam nobis
  tuæ propitiationis abundantiam multiplicatis intercessoribus, largiaris.

  \switchcolumn

  \noindent Almighty and everlasting God, who hast enabled us to honour in one
  solemn feast the merits of all thy saints: we beseech thee, that, with so many
  praying for us, thou wouldst pour forth on us the abundance of thy mercy for
  which we long.
\end{paracol}

\begin{center}\emph{Lesson: Apoc.~7:2-12.}\end{center}

\begin{paracol}{2}
  \noindent In diebus illis: Ecce, ego Ioannes vidi alterum Angelum ascendentem
  ab ortu solis, habentem signum Dei vivi: et clamavit voce magna quatuor
  Angelis, quibus datum est nocere terræ et mari, dicens: Nolite nocere terræ et
  mari neque arboribus, quoadusque signemus servos Dei nostri in frontibus
  eorum.

  Et audivi numerum signatorum, centum qua\-dra\-gin\-ta quatuor milia signati,
  ex omni tribu filiorum Israël. Ex tribu Iuda duodecim milia signati. Ex tribu
  Ruben duodecim milia signati. Ex tribu Gad duodecim milia signati. Ex tribu
  Aser duodecim milia signati. Ex tribu Nephthali duodecim milia signati. Ex
  tribu Manasse duodecim milia signati. Ex tribu Simeon duodecim milia signati.
  Ex tribu Levi duodecim milia signati. Ex tribu Issachar duodecim milia
  signati. Ex tribu Zabulon duodecim milia signati. Ex tribu Ioseph duodecim
  milia signati. Ex tribu Beniamin du\-o\-de\-cim milia signati.

  Post hæc vidi turbam magnam, quam dinumerare nemo poterat, ex omnibus gentibus
  et tribubus et populis et linguis: stantes ante thro\-num et in conspectu
  Agni, amicti stolis albis, et palmæ in manibus eorum: et clamabant voce magna,
  dicentes: Salus Deo nostro, qui sedet super thronum, et Agno. Et omnes Angeli
  stabant in circuitu throni et seniorum et quatuor animalium: et ceciderunt in
  conspectu throni in facies suas et adoraverunt Deum, dicentes: Amen.
  Benedictio et claritas et sapientia et gratiarum actio, honor et virtus et
  fortitudo Deo nostro in sæcula sæculorum. Amen.

  \switchcolumn

  \noindent In those days: Behold, I, John, saw another angel ascending from the
  rising of the sun, having the sign of the living God; and he cried with a loud
  voice to the four angels, to whom it was given to hurt the earth and the sea,
  Saying: Hurt not the earth, nor the sea, nor the trees, till we sign the
  servants of our God in their foreheads.

  And I heard the number of them that were signed, an hundred forty-four
  thousand were \linebreak signed, of every tribe of the children of Israel. Of
  the tribe of Juda, were twelve thousand signed: Of the tribe of Ruben, twelve
  thousand signed: Of the tribe of Gad, twelve thousand signed: Of the tribe of
  Aser, twelve thousand signed: Of the tribe of Nephthali, twelve thousand
  signed: Of the tribe of Manasses, twelve thousand signed: Of the tribe of
  Simeon, twelve thousand signed: Of the tribe of Levi, twelve thousand signed:
  Of the tribe of Issachar, twelve thousand signed: Of the tribe of Zabulon,
  twelve thousand signed: Of the tribe of Joseph, twelve thousand signed: Of the
  tribe of Benjamin, twelve thousand signed.

  After this I saw a great multitude, which no man could number, of all nations,
  and tribes, and peoples, and tongues, standing before the throne, and in sight
  of the Lamb, clothed with white robes, and palms in their hands: And they
  cried with a loud voice, saying: Salvation to our God, who sitteth upon the
  throne, and to the Lamb. And all the angels stood round about the throne, and
  the ancients, and the four living creatures; and they fell down before the
  throne upon their faces, and adored God, Saying: Amen. Benediction, and glory,
  and wisdom, and thanksgiving, honour, and power, and strength to our God for
  ever and ever. Amen.
\end{paracol}

\vskip10pt

\gregorioscore{gr--timete_dominum--solesmes}

\begin{translation}
  \mref{Ps.~33:10-11.} Fear the Lord, all ye his saints: for there is no want to
  them that fear him. \Vbar{}. But they that seek the Lord shall not be deprived
  of any good.
\end{translation}

\pagebreak

\gregorioscore{al--venite_ad_me--solesmes}
\begin{translation}
  \mref{Mt.~11:28.} Come to me all you that labour and are heavy laden, and I
  will refresh you.
\end{translation}

\begin{center}\emph{Gospel: Mt.~5:1-12.}\end{center}

\begin{paracol}{2}
  \noindent In illo tempore: Videns Iesus turbas, ascendit in montem, et cum
  sedisset, accesserunt ad eum discipuli eius, et aperiens os suum, docebat eos,
  dicens: Beati pauperes spiritu: quoniam ipsorum est regnum cœlorum. Beati
  mites: quoniam ipsi possidebunt terram. Beati, qui lugent: quoniam ipsi
  consolabuntur. Beati, qui esuriunt et sitiunt iustitiam: quoniam ipsi
  saturabuntur. Beati misericordes: quoniam ipsi misericordiam consequentur.
  Beati mundo corde: quoniam ipsi Deum videbunt. Beati pacifici: quoniam filii
  Dei vocabuntur. Beati, qui persecutionem patiuntur propter iustitiam: quoniam
  ipsorum est regnum cælorum. Beati estis, cum maledixerint vobis, et persecuti
  vos fuerint, et dixerint omne malum adversum vos, mentientes, propter me:
  gaudete et exsultate, quoniam merces vestra copiosa est in cœlis.

  \switchcolumn

  \noindent At that time: Jesus, seeing the multitudes, went up into a mountain,
  and when he was set down, his disciples came unto him. And opening his mouth,
  he taught them, saying: Blessed are the poor in spirit: for theirs is the
  kingdom of heaven. Blessed are the meek: for they shall possess the land. 
  Blessed are they that mourn: for they shall be comforted. Blessed are they
  that hunger and thirst after justice: for they shall have their fill. Blessed
  are the merciful: for they shall obtain mercy. Blessed are the clean of heart:
  for they shall see God. Blessed are the peacemakers: for they shall be called
  children of God. Blessed are they that suffer persecution for justice' sake:
  for theirs is the kingdom of heaven. Blessed are ye when they shall revile
  you, and persecute you, and speak all that is evil against you, untruly, for
  my sake: Be glad and rejoice, for your reward is very great in heaven. For so
  they persecuted the prophets that were before you.
\end{paracol}

\pagebreak

\gregorioscore{of--justorum_animae--solesmes}
\begin{translation}
  \mref{Sap.~3:1-3.} The souls of the just are in the hand of God, and the
  torment of malice shall not touch them: in the sight of the unwise they seemed
  to die, but they are in peace, alleluia.
\end{translation}

\begin{center}\emph{Secret}\end{center}

\begin{paracol}{2}
  \noindent Munera tibi, Domine, nostræ devotionis offerimus: quæ et pro
  cunctorum tibi grata sint honore justorum, et nobis salutaria, te miserante,
  reddantur.

  \switchcolumn

  \noindent We offer to thee, O Lord, the gifts of our devotion: may they be
  well-pleasing to thee in honour of all the just, and through thy mercy avail
  us unto salvation.
\end{paracol}

\vskip10pt

\gregorioscore{co--beati_mundo_corde--solesmes}
\begin{translation}
  \mref{Mt.~5:8-10.} Blessed are the clean of heart, for they shall see God:
  blessed are the peace-makers, for they shall be called the children of God:
  blessed are they that suffer persecution for justice' sake, for theirs is the
  kingdom of heaven.
\end{translation}

\begin{center}\emph{Postcommunion}\end{center}

\begin{paracol}{2}
  \noindent Da, quæsumus, Domine, fidelibus populis omnium Sanctorum semper
  veneratione lætari: et eorum perpetua supplicatione muniri.

  \switchcolumn

  \noindent Grant, we beseech thee, O Lord, that thy faithful people may ever
  rejoice in honouring all thy saints, and may be defended by their unceasing
  prayers.
\end{paracol}

\end{document}
