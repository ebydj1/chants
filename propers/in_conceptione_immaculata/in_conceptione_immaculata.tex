% !TEX TS-program = lualatex
% !TEX encoding = UTF-8

\documentclass[11pt]{article} % use larger type; default would be 10pt

% usual packages loading:
\usepackage{fontspec}
\usepackage{graphicx} % support the \includegraphics command and options
\usepackage{geometry} % See geometry.pdf to learn the layout options. There are lots.
\geometry{letterpaper} % or letterpaper (US) or a5paper or....
\usepackage[autocompile]{gregoriotex} % for gregorio score inclusion
\usepackage{fancyhdr}
\usepackage{fullpage}
\usepackage{scrextend}
\usepackage{paracol}

\newcommand{\mref}{\textbf}
\newcommand{\vref}{\emph}
\newenvironment{translation}
  {\begin{addmargin}{20pt}}
  {\end{addmargin}}

\pagestyle{fancy}
\fancyhf{}
\chead{\textsc{In Conceptione Immaculata B.M.V.}}
\cfoot{\thepage}
\renewcommand{\headrulewidth}{0pt}
\setlength{\headheight}{13.6pt}
\setlength{\headsep}{6pt}

\begin{document}

\thispagestyle{plain}

\begin{center}\begin{huge}
  \textsc{In Conceptione Immaculata Beatae Mariae Virginis}
\end{huge}\end{center}

\gregorioscore{in--gaudens_gaudebo_quasi--solesmes}
\begin{translation}
  \mref{Is.~61:10.} I will greatly rejoice in the Lord, and my soul shall be
  joyful in my God: for he hath clothed me with the garments of salvation, and
  with the robe of justice he hath covered me, as a bride adorned with her
  jewels. \vref{Ps.~29:2.} I will extol thee, O Lord, for thou hast upheld me:
  and hast not made my enemies to rejoice over me.
\end{translation}

\begin{center}\emph{Collect}\end{center}

\setlength{\columnseprule}{0.2pt}
\colseprulecolor{gray}

\begin{paracol}{2}
  \noindent Deus, qui per immaculatam Virginis Con\-cep\-ti\-o\-nem dignum
  Filio tuo habitaculum præparasti: quaesumus; ut, qui ex morte eiusdem Filii
  tui prævisa eam ab omni labe præservasti, nos quo\-que mundos eius
  intercessione ad te pervenire concedas.

  \switchcolumn

  \noindent O God, who by the Immaculate Conception of the Virgin didst prepare
  a fitting habitation for thy Son; grant, we beseech thee, that as thou
  preservedst her from all stain through the death foreseen of the same thy Son,
  so we too may through her intercession be purified and come to thee.
\end{paracol}

\begin{center}\emph{Lesson: Prov.~8:22-35.}\end{center}

\begin{paracol}{2}
  \noindent Dóminus possedit me in iniítio viarum suarum, antequam quidquam
  faceret a principio. Ab æ\-ter\-no ordinata sum, et ex antiquis, antequam 
  terra fieret. Nondum erant abyssi, et ego iam concepta eram: necdum fontes
  aquarum eruperant: necdum montes gravi mole constiterant: an\-te colles ego
  parturiebar: adhuc terram non fecerat et flumina et cardines orbis terræ.
  Quando præparabat coelos, aderam: quando certa lege et gyro vallabat abyssos:
  quando aethera firmabat sursum et librabat fontes aquarum: quando circumdabat
  mari terminum suum et legem ponebat aquis, ne transirent fines suos: quando
  appendebat fundamenta terræ. Cum eo eram cuncta componens: et delectabar per
  singulos dies, ludens coram eo omni tempore: ludens in orbe terrarum: et
  deliciæ meæ esse cum filiis hominum. Nunc ergo, filii, audite me: Beati, qui
  custodiunt vias meas. Audite disciplinam, et estote sapientes, et nolite
  abiicere eam. Beatus homo, qui audit me et qui vigilat ad fores meas cotidie,
  et observat ad postes ostii mei. Qui me invenerit, inveniet vitam et hauriet
  salutem a Domino.

  \switchcolumn

  \noindent The Lord possessed me in the beginning of his ways, before he made
  anything, from the beginning. I was set up from eternity, and of old, before
  the earth was made. The depths were not as yet, and I was already conceived;
  neither had the fountains of waters as yet sprung out; the mountains with
  their huge bulk had not as yet been established, before the hills I was
  brought forth. He had not yet made the earth, nor the rivers, nor the poles
  of the world. When he prepared the heavens, I was present; when with a
  certain law and compass he enclosed the depths; when he established the sky
  above, and poised the fountains of waters; when he compassed the sea with its
  bounds, and set a law to the waters that they should not pass their limits;
  when he balanced the foundations of the earth; I was with him, forming all
  things, and was delighted every day, playing before him at all times, playing
  in the world, and my delights were to be with the children of men. Now,
  therefore, ye children, hear me: blessed art they that keep my ways. Hear
  instruction, and be wise, and refuse it not. Blessed is the man that heareth
  me, and that watcheth daily at my gates and waiteth at the posts of my doors.
  He that shall find me shall find life, and shall have salvation from the Lord.
\end{paracol}

\vskip10pt

\gregorioscore{gr--benedicta_es_tu--solesmes}

\begin{translation}
  \mref{Jdt.~13:23.} Blessed art thou, O Virgin Mary, by the Lord the most
  high God, above all women upon the earth. \Vbar{}. Thou art the glory of
  Jerusalem, thou art the joy of Israel, thou art the honour of our people.
\end{translation}

\gregorioscore{al--tota_pulchra_es--solesmes}
\begin{translation}
  \mref{Cant.~4:7.} Thou art all fair, O Mary, and the stain of original sin is
  not in thee.
\end{translation}

\begin{center}\emph{Gospel: Lk.~1:26-28.}\end{center}

\begin{paracol}{2}
  \noindent In illo tempore: Missus est Angelus Gabriel a Deo in civitatem
  Galilaeæ, cui nomen Nazareth, ad Virginem desponsatam viro, cui nomen erat
  Ioseph, de domo David, et nomen Virginis Maria. Et ingressus Angelus ad eam,
  dixit: Ave, gratia plena; Dominus tecum: benedicta tu in mu\-li\-e\-ri\-bus.

  \switchcolumn

  \noindent At that time: The Angel Gabriel was sent from God into a city of
  Galilee, called Nazareth, to a virgin espoused to a man whose name was
  Joseph, of the house of David; and the virgin's name was Mary. And the Angel
  being come in, said unto her: Hail, full of grace, the Lord is with thee;
  blessed art thou among women.
\end{paracol}

\vskip20pt

\gregorioscore{of--ave_maria--solesmes}
\begin{translation}
  \mref{Lk.~1:28.} Hail, Mary, full of grace: the Lord is with thee: blessed art
  thou among women, alleluia.
\end{translation}

\pagebreak

\begin{center}\emph{Secret}\end{center}

\begin{paracol}{2}
  \noindent Salutarem hostiam, quam in sollemnitate immaculatæ Conceptionis
  beatæ Virginis Mariæ tibi, Domine, offerimus, suscipe et præsta: ut, sicut
  illam tua gratia præveniente ab omni labe immunem profitemur; ita eius
  intercessione a culpis omnibus liberemur.

  \switchcolumn

  \noindent Receive, O Lord, the saving victim which we offer to thee in
  the festival of the Immaculate Conception of the blessed Virgin Mary: and
  grant that, even as we proclaim her to have been preserved by thy grace from
  every stain, so may we be delivered, by her intercession, from all our sins.
\end{paracol}

\vskip10pt

\gregorioscore{co--gloriosa--solesmes}
\begin{translation}
  Glorious things are told of thee, O Mary, for he who is mighty hath done great
  things for thee.
\end{translation}

\begin{center}\emph{Postcommunion}\end{center}

\begin{paracol}{2}
  \noindent Sacramenta quæ sumpsimus, Domine, Deus noster: illius in nobis
  culpæ vulnera reparent; a qua immaculatam beatæ Mariæ Conceptionem
  singulariter præservasti.

  \switchcolumn

  \noindent May the sacraments which we have received, O Lord our God, heal in
  us the wounds of that sin from which thou didst by a singular privilege
  preserve the Immaculate Conception of the blessed Mary.
\end{paracol}

\end{document}
